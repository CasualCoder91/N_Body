\documentclass[tikz,border=3.14mm]{standalone}
\usepackage{tikz,tikz-3dplot} 
\usetikzlibrary{3d,shadings}
\makeatletter
% small fix for canvas is xy plane at z % https://tex.stackexchange.com/a/48776/121799
 \tikzoption{canvas is xy plane at z}[]{%
    \def\tikz@plane@origin{\pgfpointxyz{0}{0}{#1}}%
    \def\tikz@plane@x{\pgfpointxyz{1}{0}{#1}}%
    \def\tikz@plane@y{\pgfpointxyz{0}{1}{#1}}%
    \tikz@canvas@is@plane}
\makeatother

%% style for surfaces
\tikzset{surface/.style={dash pattern=on 2pt off 2pt, draw=blue!70!black, fill=gray!40!white, fill opacity=.3}}
\tikzset{surfaceR/.style={draw=blue!70!black, fill=gray!40!white, fill opacity=.6}}

\newcommand{\coneback}[4][]{
  %% start at the correct point on the circle, draw the arc, then draw to the origin of the diagram, then close the path
  \draw[canvas is xy plane at z=#2, #1] (45-#4:#3) arc (45-#4:225+#4:#3) -- (O) --cycle;
  }
\newcommand{\conefront}[4][]{
  \draw[canvas is xy plane at z=#2, #1] (45-#4:#3) arc (45-#4:-135+#4:#3) -- (O) --cycle;
  }

\begin{document}

\tdplotsetmaincoords{80}{45} % - because of difference between active and passive transformations...
\begin{tikzpicture}
 %\draw (-5,-2.5) rectangle (1.5,5);
 \coordinate (vP) at (4,2,1);
 \begin{scope}[tdplot_main_coords,thick]
  % just in case you want to get an intuition for the coordinates/projections
    % origin
    \coordinate (O) at (0,0,0);
    \draw[->] (-1,0,0) -- (6,0,0) node[right] {$x$};
    \draw[->] (0,-1,0) -- (0,6,0) node[right] {$y$};
    \draw[->] (0,0,-0.5) -- (0,0,5) node[above] {$z$};
    \draw[->] (O) -- (vP) node[midway,above left] {$\vec{t}$} node[below] {$vP$};
  % top
  \begin{scope}
\coordinate (O) at (0,0,0);
  \coneback[surface]{3}{2}{10}
  \conefront[surface]{3}{2}{10}
  \draw (0,1,1) arc(90:180:1) node[midway,below left]{$\alpha$};
  \end{scope}
  % left
  \begin{scope}[rotate=-45,shift=(vP)]
   %\draw[thick] (O) -- (0,0);
  \coordinate (O) at (0,0,0);
  \coordinate (focus) at (0,0,5);
  \coneback[surfaceR]{3}{2}{10}
    \draw[->] (O) -- (focus) node[above] {$F$};
  \conefront[surfaceR]{3}{2}{10}
    
  \end{scope}
 \end{scope}
\end{tikzpicture}
\end{document}
